% Options for packages loaded elsewhere
\PassOptionsToPackage{unicode}{hyperref}
\PassOptionsToPackage{hyphens}{url}
%
\documentclass[
]{article}
\usepackage{amsmath,amssymb}
\usepackage{iftex}
\ifPDFTeX
  \usepackage[T1]{fontenc}
  \usepackage[utf8]{inputenc}
  \usepackage{textcomp} % provide euro and other symbols
\else % if luatex or xetex
  \usepackage{unicode-math} % this also loads fontspec
  \defaultfontfeatures{Scale=MatchLowercase}
  \defaultfontfeatures[\rmfamily]{Ligatures=TeX,Scale=1}
\fi
\usepackage{lmodern}
\ifPDFTeX\else
  % xetex/luatex font selection
\fi
% Use upquote if available, for straight quotes in verbatim environments
\IfFileExists{upquote.sty}{\usepackage{upquote}}{}
\IfFileExists{microtype.sty}{% use microtype if available
  \usepackage[]{microtype}
  \UseMicrotypeSet[protrusion]{basicmath} % disable protrusion for tt fonts
}{}
\makeatletter
\@ifundefined{KOMAClassName}{% if non-KOMA class
  \IfFileExists{parskip.sty}{%
    \usepackage{parskip}
  }{% else
    \setlength{\parindent}{0pt}
    \setlength{\parskip}{6pt plus 2pt minus 1pt}}
}{% if KOMA class
  \KOMAoptions{parskip=half}}
\makeatother
\usepackage{xcolor}
\usepackage[margin=1in]{geometry}
\usepackage{color}
\usepackage{fancyvrb}
\newcommand{\VerbBar}{|}
\newcommand{\VERB}{\Verb[commandchars=\\\{\}]}
\DefineVerbatimEnvironment{Highlighting}{Verbatim}{commandchars=\\\{\}}
% Add ',fontsize=\small' for more characters per line
\usepackage{framed}
\definecolor{shadecolor}{RGB}{248,248,248}
\newenvironment{Shaded}{\begin{snugshade}}{\end{snugshade}}
\newcommand{\AlertTok}[1]{\textcolor[rgb]{0.94,0.16,0.16}{#1}}
\newcommand{\AnnotationTok}[1]{\textcolor[rgb]{0.56,0.35,0.01}{\textbf{\textit{#1}}}}
\newcommand{\AttributeTok}[1]{\textcolor[rgb]{0.13,0.29,0.53}{#1}}
\newcommand{\BaseNTok}[1]{\textcolor[rgb]{0.00,0.00,0.81}{#1}}
\newcommand{\BuiltInTok}[1]{#1}
\newcommand{\CharTok}[1]{\textcolor[rgb]{0.31,0.60,0.02}{#1}}
\newcommand{\CommentTok}[1]{\textcolor[rgb]{0.56,0.35,0.01}{\textit{#1}}}
\newcommand{\CommentVarTok}[1]{\textcolor[rgb]{0.56,0.35,0.01}{\textbf{\textit{#1}}}}
\newcommand{\ConstantTok}[1]{\textcolor[rgb]{0.56,0.35,0.01}{#1}}
\newcommand{\ControlFlowTok}[1]{\textcolor[rgb]{0.13,0.29,0.53}{\textbf{#1}}}
\newcommand{\DataTypeTok}[1]{\textcolor[rgb]{0.13,0.29,0.53}{#1}}
\newcommand{\DecValTok}[1]{\textcolor[rgb]{0.00,0.00,0.81}{#1}}
\newcommand{\DocumentationTok}[1]{\textcolor[rgb]{0.56,0.35,0.01}{\textbf{\textit{#1}}}}
\newcommand{\ErrorTok}[1]{\textcolor[rgb]{0.64,0.00,0.00}{\textbf{#1}}}
\newcommand{\ExtensionTok}[1]{#1}
\newcommand{\FloatTok}[1]{\textcolor[rgb]{0.00,0.00,0.81}{#1}}
\newcommand{\FunctionTok}[1]{\textcolor[rgb]{0.13,0.29,0.53}{\textbf{#1}}}
\newcommand{\ImportTok}[1]{#1}
\newcommand{\InformationTok}[1]{\textcolor[rgb]{0.56,0.35,0.01}{\textbf{\textit{#1}}}}
\newcommand{\KeywordTok}[1]{\textcolor[rgb]{0.13,0.29,0.53}{\textbf{#1}}}
\newcommand{\NormalTok}[1]{#1}
\newcommand{\OperatorTok}[1]{\textcolor[rgb]{0.81,0.36,0.00}{\textbf{#1}}}
\newcommand{\OtherTok}[1]{\textcolor[rgb]{0.56,0.35,0.01}{#1}}
\newcommand{\PreprocessorTok}[1]{\textcolor[rgb]{0.56,0.35,0.01}{\textit{#1}}}
\newcommand{\RegionMarkerTok}[1]{#1}
\newcommand{\SpecialCharTok}[1]{\textcolor[rgb]{0.81,0.36,0.00}{\textbf{#1}}}
\newcommand{\SpecialStringTok}[1]{\textcolor[rgb]{0.31,0.60,0.02}{#1}}
\newcommand{\StringTok}[1]{\textcolor[rgb]{0.31,0.60,0.02}{#1}}
\newcommand{\VariableTok}[1]{\textcolor[rgb]{0.00,0.00,0.00}{#1}}
\newcommand{\VerbatimStringTok}[1]{\textcolor[rgb]{0.31,0.60,0.02}{#1}}
\newcommand{\WarningTok}[1]{\textcolor[rgb]{0.56,0.35,0.01}{\textbf{\textit{#1}}}}
\usepackage{longtable,booktabs,array}
\usepackage{calc} % for calculating minipage widths
% Correct order of tables after \paragraph or \subparagraph
\usepackage{etoolbox}
\makeatletter
\patchcmd\longtable{\par}{\if@noskipsec\mbox{}\fi\par}{}{}
\makeatother
% Allow footnotes in longtable head/foot
\IfFileExists{footnotehyper.sty}{\usepackage{footnotehyper}}{\usepackage{footnote}}
\makesavenoteenv{longtable}
\usepackage{graphicx}
\makeatletter
\def\maxwidth{\ifdim\Gin@nat@width>\linewidth\linewidth\else\Gin@nat@width\fi}
\def\maxheight{\ifdim\Gin@nat@height>\textheight\textheight\else\Gin@nat@height\fi}
\makeatother
% Scale images if necessary, so that they will not overflow the page
% margins by default, and it is still possible to overwrite the defaults
% using explicit options in \includegraphics[width, height, ...]{}
\setkeys{Gin}{width=\maxwidth,height=\maxheight,keepaspectratio}
% Set default figure placement to htbp
\makeatletter
\def\fps@figure{htbp}
\makeatother
\setlength{\emergencystretch}{3em} % prevent overfull lines
\providecommand{\tightlist}{%
  \setlength{\itemsep}{0pt}\setlength{\parskip}{0pt}}
\setcounter{secnumdepth}{-\maxdimen} % remove section numbering
\ifLuaTeX
  \usepackage{selnolig}  % disable illegal ligatures
\fi
\usepackage{bookmark}
\IfFileExists{xurl.sty}{\usepackage{xurl}}{} % add URL line breaks if available
\urlstyle{same}
\hypersetup{
  pdftitle={Rapport},
  pdfauthor={Ahamad MOHAMMAD; Minko Bikono NEIL-JOVY; Simon GELBART; Willen AMICHE},
  hidelinks,
  pdfcreator={LaTeX via pandoc}}

\title{Rapport}
\author{Ahamad MOHAMMAD; Minko Bikono NEIL-JOVY; Simon GELBART; Willen
AMICHE}
\date{2025-04-25}

\begin{document}
\maketitle

\subsection{\texorpdfstring{\textbf{Introduction}}{Introduction}}\label{introduction}

Notre objectif est d'explorer l'impact des stratégies de recrutement sur
les performances sportives des clubs et joueurs. En combinant des
statistiques individuelles, collectives et des données de transferts,
nous chercherons à identifier les tendances qui influencent la réussite
des équipes sur plusieurs saisons.

\subsubsection{\texorpdfstring{\textbf{📊 Description des
variables}}{📊 Description des variables}}\label{description-des-variables}

\paragraph{\texorpdfstring{- 🧍‍♂️ Player Stats 2021-2022
(\texttt{2021-2022\ Football\ Player\ Stats.csv})}{- 🧍‍♂️ Player Stats 2021-2022 (2021-2022 Football Player Stats.csv)}}\label{player-stats-2021-2022-2021-2022-football-player-stats.csv}

143 variables -- chaque ligne correspond à un joueur pour la saison
2021-2022.

\begin{longtable}[]{@{}
  >{\raggedright\arraybackslash}p{(\columnwidth - 4\tabcolsep) * \real{0.3276}}
  >{\raggedright\arraybackslash}p{(\columnwidth - 4\tabcolsep) * \real{0.1897}}
  >{\raggedright\arraybackslash}p{(\columnwidth - 4\tabcolsep) * \real{0.4828}}@{}}
\toprule\noalign{}
\begin{minipage}[b]{\linewidth}\raggedright
Variable
\end{minipage} & \begin{minipage}[b]{\linewidth}\raggedright
Type
\end{minipage} & \begin{minipage}[b]{\linewidth}\raggedright
Description approximative
\end{minipage} \\
\midrule\noalign{}
\endhead
\bottomrule\noalign{}
\endlastfoot
\texttt{Rk} & int64 & Rang ou ID du joueur \\
\texttt{Player} & object & Nom du joueur \\
\texttt{Nation} & object & Nationalité \\
\texttt{Pos} & object & Poste \\
\texttt{Squad} & object & Club \\
\texttt{Comp} & object & Compétition principale \\
\texttt{Age} & object & Âge \\
\texttt{Born} & object & Année de naissance \\
\ldots{} & \ldots{} & \ldots et plus de 130 autres statistiques de jeu :
buts, passes, tirs, dribbles, fautes, tacles, interceptions, passes
progressives, etc. \\
\texttt{AerWon} & float64 & Duels aériens gagnés \\
\texttt{AerLost} & float64 & Duels aériens perdus \\
\texttt{AerWon\%} & float64 & Pourcentage de duels aériens gagnés \\
\end{longtable}

\begin{center}\rule{0.5\linewidth}{0.5pt}\end{center}

\paragraph{\texorpdfstring{- 🏟️ Team Stats 2021-2022
(\texttt{2021-2022\ Football\ Team\ Stats.csv})}{- 🏟️ Team Stats 2021-2022 (2021-2022 Football Team Stats.csv)}}\label{team-stats-2021-2022-2021-2022-football-team-stats.csv}

20 variables -- chaque ligne correspond à une équipe.

\begin{longtable}[]{@{}
  >{\raggedright\arraybackslash}p{(\columnwidth - 4\tabcolsep) * \real{0.4286}}
  >{\raggedright\arraybackslash}p{(\columnwidth - 4\tabcolsep) * \real{0.2619}}
  >{\raggedright\arraybackslash}p{(\columnwidth - 4\tabcolsep) * \real{0.3095}}@{}}
\toprule\noalign{}
\begin{minipage}[b]{\linewidth}\raggedright
Variable
\end{minipage} & \begin{minipage}[b]{\linewidth}\raggedright
Type
\end{minipage} & \begin{minipage}[b]{\linewidth}\raggedright
Description
\end{minipage} \\
\midrule\noalign{}
\endhead
\bottomrule\noalign{}
\endlastfoot
\texttt{Rk} & int64 & Rang \\
\texttt{Squad} & object & Nom du club \\
\texttt{Country} & object & Pays \\
\texttt{LgRk} & int64 & Classement dans la ligue \\
\texttt{MP}, \texttt{W}, \texttt{D}, \texttt{L} & int64 & Matchs joués,
Victoires, Nuls, Défaites \\
\texttt{GF}, \texttt{GA}, \texttt{GD} & int64 & Buts pour, contre,
différence \\
\texttt{Pts}, \texttt{Pts/G} & int64 / float64 & Points et moyenne par
match \\
\texttt{xG}, \texttt{xGA}, \texttt{xGD}, \texttt{xGD/90} & float64 &
Données d'expected goals \\
\texttt{Attendance} & int64 & Affluence moyenne \\
\texttt{Top\ Team\ Scorer} & object & Meilleur buteur \\
\texttt{Goalkeeper} & object & Gardien principal \\
\end{longtable}

\begin{center}\rule{0.5\linewidth}{0.5pt}\end{center}

\paragraph{\texorpdfstring{- 🏟️ Team Stats 2022-2023
(\texttt{2022-2023\ Football\ Team\ Stats.csv})}{- 🏟️ Team Stats 2022-2023 (2022-2023 Football Team Stats.csv)}}\label{team-stats-2022-2023-2022-2023-football-team-stats.csv}

Même structure et signification que pour 2021-2022, mais avec la saison
suivante.

\begin{center}\rule{0.5\linewidth}{0.5pt}\end{center}

\paragraph{\texorpdfstring{- 🔄 Transfers Été 2022
(\texttt{2022\_2023\_football\_summer\_transfers.csv})}{- 🔄 Transfers Été 2022 (2022\_2023\_football\_summer\_transfers.csv)}}\label{transfers-uxe9tuxe9-2022-2022_2023_football_summer_transfers.csv}

11 variables -- chaque ligne correspond à un transfert.

\begin{longtable}[]{@{}
  >{\raggedright\arraybackslash}p{(\columnwidth - 4\tabcolsep) * \real{0.4211}}
  >{\raggedright\arraybackslash}p{(\columnwidth - 4\tabcolsep) * \real{0.2368}}
  >{\raggedright\arraybackslash}p{(\columnwidth - 4\tabcolsep) * \real{0.3421}}@{}}
\toprule\noalign{}
\begin{minipage}[b]{\linewidth}\raggedright
Variable
\end{minipage} & \begin{minipage}[b]{\linewidth}\raggedright
Type
\end{minipage} & \begin{minipage}[b]{\linewidth}\raggedright
Description
\end{minipage} \\
\midrule\noalign{}
\endhead
\bottomrule\noalign{}
\endlastfoot
\texttt{name} & object & Nom du joueur transféré \\
\texttt{position} & object & Poste \\
\texttt{age} & object & Âge \\
\texttt{market\_value} & object & Valeur estimée \\
\texttt{country\_from} & object & Pays de départ \\
\texttt{league\_from} & object & Ligue de départ \\
\texttt{club\_from} & object & Club de départ \\
\texttt{country\_to} & object & Pays d'arrivée \\
\texttt{league\_to} & object & Ligue d'arrivée \\
\texttt{club\_to} & object & Club d'arrivée \\
\texttt{fee} & object & Montant du transfert (peut contenir ``Free'',
``Loan'', etc.) \\
\end{longtable}

\begin{center}\rule{0.5\linewidth}{0.5pt}\end{center}

\subsection{\texorpdfstring{\textbf{Analyse et réponses aux
questions}}{Analyse et réponses aux questions}}\label{analyse-et-ruxe9ponses-aux-questions}

\begin{Shaded}
\begin{Highlighting}[]
\CommentTok{\# Chargement des packages nécessaires}

\FunctionTok{library}\NormalTok{(tidyverse)}
\end{Highlighting}
\end{Shaded}

\begin{verbatim}
## Warning: le package 'tidyverse' a été compilé avec la version R 4.3.3
\end{verbatim}

\begin{verbatim}
## Warning: le package 'ggplot2' a été compilé avec la version R 4.3.3
\end{verbatim}

\begin{verbatim}
## Warning: le package 'tibble' a été compilé avec la version R 4.3.3
\end{verbatim}

\begin{verbatim}
## Warning: le package 'tidyr' a été compilé avec la version R 4.3.3
\end{verbatim}

\begin{verbatim}
## Warning: le package 'readr' a été compilé avec la version R 4.3.3
\end{verbatim}

\begin{verbatim}
## Warning: le package 'purrr' a été compilé avec la version R 4.3.3
\end{verbatim}

\begin{verbatim}
## Warning: le package 'dplyr' a été compilé avec la version R 4.3.3
\end{verbatim}

\begin{verbatim}
## Warning: le package 'forcats' a été compilé avec la version R 4.3.3
\end{verbatim}

\begin{verbatim}
## Warning: le package 'lubridate' a été compilé avec la version R 4.3.3
\end{verbatim}

\begin{Shaded}
\begin{Highlighting}[]
\FunctionTok{library}\NormalTok{(ggplot2)}
\end{Highlighting}
\end{Shaded}

\begin{Shaded}
\begin{Highlighting}[]
\CommentTok{\# Importation des données}

\NormalTok{player\_stats\_2021\_2022 }\OtherTok{\textless{}{-}} \FunctionTok{read.csv}\NormalTok{(}\StringTok{"data/2021{-}2022 Football Player Stats.csv"}\NormalTok{, }\AttributeTok{sep =} \StringTok{";"}\NormalTok{, }\AttributeTok{fileEncoding =} \StringTok{"ISO{-}8859{-}1"}\NormalTok{)}
\NormalTok{team\_stats\_2021\_2022 }\OtherTok{\textless{}{-}} \FunctionTok{read.csv}\NormalTok{(}\StringTok{"data/2021{-}2022 Football Team Stats.csv"}\NormalTok{, }\AttributeTok{sep =} \StringTok{";"}\NormalTok{, }\AttributeTok{fileEncoding =} \StringTok{"ISO{-}8859{-}1"}\NormalTok{)}
\NormalTok{team\_stats\_2022\_2023 }\OtherTok{\textless{}{-}} \FunctionTok{read.csv}\NormalTok{(}\StringTok{"data/2022{-}2023 Football Team Stats.csv"}\NormalTok{, }\AttributeTok{sep =} \StringTok{";"}\NormalTok{, }\AttributeTok{fileEncoding =} \StringTok{"ISO{-}8859{-}1"}\NormalTok{)}
\NormalTok{transfers\_2022 }\OtherTok{\textless{}{-}} \FunctionTok{read.csv}\NormalTok{(}\StringTok{"data/2022\_2023\_football\_summer\_transfers.csv"}\NormalTok{, }\AttributeTok{sep =} \StringTok{","}\NormalTok{, }\AttributeTok{fileEncoding =} \StringTok{"ISO{-}8859{-}1"}\NormalTok{)}
\end{Highlighting}
\end{Shaded}

\begin{center}\rule{0.5\linewidth}{0.5pt}\end{center}

\subsubsection{\texorpdfstring{\textbf{Question 1:} Quels clubs ont le
plus recruté par poste (top 20)
?}{Question 1: Quels clubs ont le plus recruté par poste (top 20) ?}}\label{question-1-quels-clubs-ont-le-plus-recrutuxe9-par-poste-top-20}

\paragraph{🎯 Objectif:}\label{objectif}

Identifier les \textbf{20 clubs les plus actifs} lors du mercato d'été
2022 en termes de nombre de recrues, puis visualiser quels postes ont
été ciblés en priorité par ces clubs. Cela permet de mieux comprendre
les \textbf{stratégies de renforcement} des effectifs selon les lignes
de jeu (défense, milieu, attaque).

\begin{Shaded}
\begin{Highlighting}[]
\CommentTok{\# Code}

\CommentTok{\# Nettoyage de base}
\NormalTok{transfers\_clean }\OtherTok{\textless{}{-}}\NormalTok{ transfers\_2022 }\SpecialCharTok{\%\textgreater{}\%}
  \FunctionTok{filter}\NormalTok{(}\SpecialCharTok{!}\FunctionTok{is.na}\NormalTok{(club\_to), }\SpecialCharTok{!}\FunctionTok{is.na}\NormalTok{(position), position }\SpecialCharTok{!=} \StringTok{""}\NormalTok{)}

\CommentTok{\# Regrouper et compter}
\NormalTok{recrutements\_par\_poste }\OtherTok{\textless{}{-}}\NormalTok{ transfers\_clean }\SpecialCharTok{\%\textgreater{}\%}
  \FunctionTok{group\_by}\NormalTok{(club\_to, position) }\SpecialCharTok{\%\textgreater{}\%}
  \FunctionTok{summarise}\NormalTok{(}\AttributeTok{nb\_recrues =} \FunctionTok{n}\NormalTok{(), }\AttributeTok{.groups =} \StringTok{"drop"}\NormalTok{)}

\CommentTok{\# Garder uniquement les 20 clubs ayant recruté le plus globalement}
\NormalTok{top\_clubs }\OtherTok{\textless{}{-}}\NormalTok{ recrutements\_par\_poste }\SpecialCharTok{\%\textgreater{}\%}
  \FunctionTok{group\_by}\NormalTok{(club\_to) }\SpecialCharTok{\%\textgreater{}\%}
  \FunctionTok{summarise}\NormalTok{(}\AttributeTok{total\_recrues =} \FunctionTok{sum}\NormalTok{(nb\_recrues)) }\SpecialCharTok{\%\textgreater{}\%}
  \FunctionTok{top\_n}\NormalTok{(}\DecValTok{20}\NormalTok{, total\_recrues) }\SpecialCharTok{\%\textgreater{}\%}
  \FunctionTok{pull}\NormalTok{(club\_to)}

\CommentTok{\# Filtrer les données}
\NormalTok{recrutements\_top }\OtherTok{\textless{}{-}}\NormalTok{ recrutements\_par\_poste }\SpecialCharTok{\%\textgreater{}\%}
  \FunctionTok{filter}\NormalTok{(club\_to }\SpecialCharTok{\%in\%}\NormalTok{ top\_clubs)}

\CommentTok{\# Visualisation }
\FunctionTok{ggplot}\NormalTok{(recrutements\_top, }\FunctionTok{aes}\NormalTok{(}\AttributeTok{x =}\NormalTok{ position, }\AttributeTok{y =} \FunctionTok{fct\_reorder}\NormalTok{(club\_to, nb\_recrues), }\AttributeTok{fill =}\NormalTok{ nb\_recrues)) }\SpecialCharTok{+}
  \FunctionTok{geom\_tile}\NormalTok{(}\AttributeTok{color =} \StringTok{"white"}\NormalTok{) }\SpecialCharTok{+}
  \FunctionTok{scale\_fill\_gradient}\NormalTok{(}\AttributeTok{low =} \StringTok{"lightblue"}\NormalTok{, }\AttributeTok{high =} \StringTok{"darkblue"}\NormalTok{) }\SpecialCharTok{+}
  \FunctionTok{labs}\NormalTok{(}
    \AttributeTok{title =} \StringTok{"Top 20 clubs – nombre de recrues par poste (été 2022)"}\NormalTok{,}
    \AttributeTok{x =} \StringTok{"Poste"}\NormalTok{,}
    \AttributeTok{y =} \StringTok{"Club"}\NormalTok{,}
    \AttributeTok{fill =} \StringTok{"Nombre de recrues"}
\NormalTok{  ) }\SpecialCharTok{+}
  \FunctionTok{theme\_minimal}\NormalTok{(}\AttributeTok{base\_size =} \DecValTok{12}\NormalTok{) }\SpecialCharTok{+}
  \FunctionTok{theme}\NormalTok{(}\AttributeTok{axis.text.x =} \FunctionTok{element\_text}\NormalTok{(}\AttributeTok{angle =} \DecValTok{45}\NormalTok{, }\AttributeTok{hjust =} \DecValTok{1}\NormalTok{))}
\end{Highlighting}
\end{Shaded}

\includegraphics{rapport_files/figure-latex/unnamed-chunk-1-1.pdf}

\paragraph{🧠 Interprétation du
graphique:}\label{interpruxe9tation-du-graphique}

\begin{itemize}
\item
  🔵 \textbf{US Salernitana 1919} est le club qui a le plus recruté tous
  postes confondus, avec un \textbf{focus important sur les milieux
  offensifs} et \textbf{défenseurs centraux}.
\item
  🔵 \textbf{US Lecce}, \textbf{US Cremonese} et \textbf{Udinese Calcio}
  (clubs italiens) montrent aussi une stratégie de \textbf{renforcement
  défensif}, particulièrement en \textbf{centre-back}.
\item
  🔵 \textbf{Olympique de Marseille} et \textbf{Nottingham Forest} ont
  \textbf{diversifié leurs recrutements} sur plusieurs lignes, y compris
  \textbf{les ailes} (\emph{left/right winger}).
\item
  🎯 On observe une \textbf{forte demande en milieux de terrain},
  notamment :

  \begin{itemize}
  \tightlist
  \item
    \emph{Attacking Midfield}
  \item
    \emph{Defensive Midfield}
  \item
    \emph{Centre Midfield}
  \end{itemize}
\item
  Peu de clubs ont recruté plusieurs \textbf{gardiens}, ce qui est
  logique : un club n'en fait souvent venir qu'un seul par saison.
\item
  Certains clubs comme \textbf{FC Empoli} ou \textbf{OGC Nice}
  présentent une stratégie de recrutement \textbf{équilibrée sur
  différentes lignes}, ce qui pourrait indiquer un
  \textbf{renouvellement global de l'effectif}.
\end{itemize}

\begin{center}\rule{0.5\linewidth}{0.5pt}\end{center}

\subsubsection{\texorpdfstring{\textbf{Question 2:} Quels postes sont
les plus valorisés sur le marché
?}{Question 2: Quels postes sont les plus valorisés sur le marché ?}}\label{question-2-quels-postes-sont-les-plus-valorisuxe9s-sur-le-marchuxe9}

\paragraph{🎯 Objectif:}\label{objectif-1}

Visualiser la valeur marchande des joueurs par poste pour comprendre
quels types de profils sont les plus prisés financièrement sur le
marché. Cela permet de hiérarchiser les postes selon leur importance
économique dans le football professionnel.

\begin{Shaded}
\begin{Highlighting}[]
\CommentTok{\# Nettoyage de base}
\NormalTok{transfers\_clean }\OtherTok{\textless{}{-}}\NormalTok{ transfers\_2022 }\SpecialCharTok{\%\textgreater{}\%}
  \FunctionTok{filter}\NormalTok{(}\SpecialCharTok{!}\FunctionTok{is.na}\NormalTok{(position), }\SpecialCharTok{!}\FunctionTok{is.na}\NormalTok{(market\_value))}

\CommentTok{\# Visualisation avec boxplot}
\FunctionTok{ggplot}\NormalTok{(transfers\_clean, }\FunctionTok{aes}\NormalTok{(}\AttributeTok{x =} \FunctionTok{reorder}\NormalTok{(position, market\_value, }\AttributeTok{FUN =}\NormalTok{ median), }\AttributeTok{y =}\NormalTok{ market\_value)) }\SpecialCharTok{+}
  \FunctionTok{geom\_boxplot}\NormalTok{(}\AttributeTok{fill =} \StringTok{"\#FFB347"}\NormalTok{, }\AttributeTok{color =} \StringTok{"black"}\NormalTok{, }\AttributeTok{outlier.color =} \StringTok{"red"}\NormalTok{, }\AttributeTok{outlier.size =} \DecValTok{1}\NormalTok{) }\SpecialCharTok{+}
  \FunctionTok{coord\_flip}\NormalTok{() }\SpecialCharTok{+}
  \FunctionTok{labs}\NormalTok{(}
    \AttributeTok{title =} \StringTok{"Valeur marchande des joueurs par poste (été 2022)"}\NormalTok{,}
    \AttributeTok{x =} \StringTok{"Poste"}\NormalTok{,}
    \AttributeTok{y =} \StringTok{"Valeur marchande (€)"}
\NormalTok{  ) }\SpecialCharTok{+}
  \FunctionTok{theme\_minimal}\NormalTok{(}\AttributeTok{base\_size =} \DecValTok{12}\NormalTok{)}
\end{Highlighting}
\end{Shaded}

\begin{verbatim}
## Warning in mean.default(sort(x, partial = half + 0L:1L)[half + 0L:1L]):
## l'argument n'est ni numérique, ni logique : renvoi de NA

## Warning in mean.default(sort(x, partial = half + 0L:1L)[half + 0L:1L]):
## l'argument n'est ni numérique, ni logique : renvoi de NA

## Warning in mean.default(sort(x, partial = half + 0L:1L)[half + 0L:1L]):
## l'argument n'est ni numérique, ni logique : renvoi de NA

## Warning in mean.default(sort(x, partial = half + 0L:1L)[half + 0L:1L]):
## l'argument n'est ni numérique, ni logique : renvoi de NA

## Warning in mean.default(sort(x, partial = half + 0L:1L)[half + 0L:1L]):
## l'argument n'est ni numérique, ni logique : renvoi de NA
\end{verbatim}

\includegraphics{rapport_files/figure-latex/unnamed-chunk-2-1.pdf}

\paragraph{🧠 Interprétation du
graphique:}\label{interpruxe9tation-du-graphique-1}

\begin{center}\rule{0.5\linewidth}{0.5pt}\end{center}

\subsubsection{\texorpdfstring{\textbf{Question
3:}}{Question 3:}}\label{question-3}

\paragraph{🎯 Objectif:}\label{objectif-2}

\begin{Shaded}
\begin{Highlighting}[]
\CommentTok{\# Code}
\end{Highlighting}
\end{Shaded}

\paragraph{🧠 Interprétation du
graphique:}\label{interpruxe9tation-du-graphique-2}

\begin{center}\rule{0.5\linewidth}{0.5pt}\end{center}

\subsubsection{\texorpdfstring{\textbf{Question
4:}}{Question 4:}}\label{question-4}

\paragraph{🎯 Objectif:}\label{objectif-3}

\begin{Shaded}
\begin{Highlighting}[]
\CommentTok{\# Code}
\end{Highlighting}
\end{Shaded}

\paragraph{🧠 Interprétation du
graphique:}\label{interpruxe9tation-du-graphique-3}

\end{document}
