% Options for packages loaded elsewhere
\PassOptionsToPackage{unicode}{hyperref}
\PassOptionsToPackage{hyphens}{url}
%
\documentclass[
]{article}
\usepackage{amsmath,amssymb}
\usepackage{iftex}
\ifPDFTeX
  \usepackage[T1]{fontenc}
  \usepackage[utf8]{inputenc}
  \usepackage{textcomp} % provide euro and other symbols
\else % if luatex or xetex
  \usepackage{unicode-math} % this also loads fontspec
  \defaultfontfeatures{Scale=MatchLowercase}
  \defaultfontfeatures[\rmfamily]{Ligatures=TeX,Scale=1}
\fi
\usepackage{lmodern}
\ifPDFTeX\else
  % xetex/luatex font selection
\fi
% Use upquote if available, for straight quotes in verbatim environments
\IfFileExists{upquote.sty}{\usepackage{upquote}}{}
\IfFileExists{microtype.sty}{% use microtype if available
  \usepackage[]{microtype}
  \UseMicrotypeSet[protrusion]{basicmath} % disable protrusion for tt fonts
}{}
\makeatletter
\@ifundefined{KOMAClassName}{% if non-KOMA class
  \IfFileExists{parskip.sty}{%
    \usepackage{parskip}
  }{% else
    \setlength{\parindent}{0pt}
    \setlength{\parskip}{6pt plus 2pt minus 1pt}}
}{% if KOMA class
  \KOMAoptions{parskip=half}}
\makeatother
\usepackage{xcolor}
\usepackage[margin=1in]{geometry}
\usepackage{color}
\usepackage{fancyvrb}
\newcommand{\VerbBar}{|}
\newcommand{\VERB}{\Verb[commandchars=\\\{\}]}
\DefineVerbatimEnvironment{Highlighting}{Verbatim}{commandchars=\\\{\}}
% Add ',fontsize=\small' for more characters per line
\usepackage{framed}
\definecolor{shadecolor}{RGB}{248,248,248}
\newenvironment{Shaded}{\begin{snugshade}}{\end{snugshade}}
\newcommand{\AlertTok}[1]{\textcolor[rgb]{0.94,0.16,0.16}{#1}}
\newcommand{\AnnotationTok}[1]{\textcolor[rgb]{0.56,0.35,0.01}{\textbf{\textit{#1}}}}
\newcommand{\AttributeTok}[1]{\textcolor[rgb]{0.13,0.29,0.53}{#1}}
\newcommand{\BaseNTok}[1]{\textcolor[rgb]{0.00,0.00,0.81}{#1}}
\newcommand{\BuiltInTok}[1]{#1}
\newcommand{\CharTok}[1]{\textcolor[rgb]{0.31,0.60,0.02}{#1}}
\newcommand{\CommentTok}[1]{\textcolor[rgb]{0.56,0.35,0.01}{\textit{#1}}}
\newcommand{\CommentVarTok}[1]{\textcolor[rgb]{0.56,0.35,0.01}{\textbf{\textit{#1}}}}
\newcommand{\ConstantTok}[1]{\textcolor[rgb]{0.56,0.35,0.01}{#1}}
\newcommand{\ControlFlowTok}[1]{\textcolor[rgb]{0.13,0.29,0.53}{\textbf{#1}}}
\newcommand{\DataTypeTok}[1]{\textcolor[rgb]{0.13,0.29,0.53}{#1}}
\newcommand{\DecValTok}[1]{\textcolor[rgb]{0.00,0.00,0.81}{#1}}
\newcommand{\DocumentationTok}[1]{\textcolor[rgb]{0.56,0.35,0.01}{\textbf{\textit{#1}}}}
\newcommand{\ErrorTok}[1]{\textcolor[rgb]{0.64,0.00,0.00}{\textbf{#1}}}
\newcommand{\ExtensionTok}[1]{#1}
\newcommand{\FloatTok}[1]{\textcolor[rgb]{0.00,0.00,0.81}{#1}}
\newcommand{\FunctionTok}[1]{\textcolor[rgb]{0.13,0.29,0.53}{\textbf{#1}}}
\newcommand{\ImportTok}[1]{#1}
\newcommand{\InformationTok}[1]{\textcolor[rgb]{0.56,0.35,0.01}{\textbf{\textit{#1}}}}
\newcommand{\KeywordTok}[1]{\textcolor[rgb]{0.13,0.29,0.53}{\textbf{#1}}}
\newcommand{\NormalTok}[1]{#1}
\newcommand{\OperatorTok}[1]{\textcolor[rgb]{0.81,0.36,0.00}{\textbf{#1}}}
\newcommand{\OtherTok}[1]{\textcolor[rgb]{0.56,0.35,0.01}{#1}}
\newcommand{\PreprocessorTok}[1]{\textcolor[rgb]{0.56,0.35,0.01}{\textit{#1}}}
\newcommand{\RegionMarkerTok}[1]{#1}
\newcommand{\SpecialCharTok}[1]{\textcolor[rgb]{0.81,0.36,0.00}{\textbf{#1}}}
\newcommand{\SpecialStringTok}[1]{\textcolor[rgb]{0.31,0.60,0.02}{#1}}
\newcommand{\StringTok}[1]{\textcolor[rgb]{0.31,0.60,0.02}{#1}}
\newcommand{\VariableTok}[1]{\textcolor[rgb]{0.00,0.00,0.00}{#1}}
\newcommand{\VerbatimStringTok}[1]{\textcolor[rgb]{0.31,0.60,0.02}{#1}}
\newcommand{\WarningTok}[1]{\textcolor[rgb]{0.56,0.35,0.01}{\textbf{\textit{#1}}}}
\usepackage{longtable,booktabs,array}
\usepackage{calc} % for calculating minipage widths
% Correct order of tables after \paragraph or \subparagraph
\usepackage{etoolbox}
\makeatletter
\patchcmd\longtable{\par}{\if@noskipsec\mbox{}\fi\par}{}{}
\makeatother
% Allow footnotes in longtable head/foot
\IfFileExists{footnotehyper.sty}{\usepackage{footnotehyper}}{\usepackage{footnote}}
\makesavenoteenv{longtable}
\usepackage{graphicx}
\makeatletter
\def\maxwidth{\ifdim\Gin@nat@width>\linewidth\linewidth\else\Gin@nat@width\fi}
\def\maxheight{\ifdim\Gin@nat@height>\textheight\textheight\else\Gin@nat@height\fi}
\makeatother
% Scale images if necessary, so that they will not overflow the page
% margins by default, and it is still possible to overwrite the defaults
% using explicit options in \includegraphics[width, height, ...]{}
\setkeys{Gin}{width=\maxwidth,height=\maxheight,keepaspectratio}
% Set default figure placement to htbp
\makeatletter
\def\fps@figure{htbp}
\makeatother
\setlength{\emergencystretch}{3em} % prevent overfull lines
\providecommand{\tightlist}{%
  \setlength{\itemsep}{0pt}\setlength{\parskip}{0pt}}
\setcounter{secnumdepth}{-\maxdimen} % remove section numbering
\ifLuaTeX
  \usepackage{selnolig}  % disable illegal ligatures
\fi
\usepackage{bookmark}
\IfFileExists{xurl.sty}{\usepackage{xurl}}{} % add URL line breaks if available
\urlstyle{same}
\hypersetup{
  pdftitle={Rapport},
  pdfauthor={Ahamad MOHAMMAD; Minko Bikono NEIL-JOVY; Simon GELBART; Willen AMICHE},
  hidelinks,
  pdfcreator={LaTeX via pandoc}}

\title{Rapport}
\author{Ahamad MOHAMMAD; Minko Bikono NEIL-JOVY; Simon GELBART; Willen
AMICHE}
\date{2025-04-27}

\begin{document}
\maketitle

\subsection{\texorpdfstring{\textbf{Introduction}}{Introduction}}\label{introduction}

Notre objectif est d'explorer l'impact des stratégies de recrutement sur
les performances sportives des clubs et joueurs. En combinant des
statistiques individuelles, collectives et des données de transferts,
nous chercherons à identifier les tendances qui influencent la réussite
des équipes sur plusieurs saisons.

\subsubsection{\texorpdfstring{\textbf{📊 Description des
variables}}{📊 Description des variables}}\label{description-des-variables}

\paragraph{\texorpdfstring{- 🧍‍♂️ Player Stats 2021-2022
(\texttt{2021-2022\ Football\ Player\ Stats.csv})}{- 🧍‍♂️ Player Stats 2021-2022 (2021-2022 Football Player Stats.csv)}}\label{player-stats-2021-2022-2021-2022-football-player-stats.csv}

143 variables -- chaque ligne correspond à un joueur pour la saison
2021-2022.

\begin{longtable}[]{@{}
  >{\raggedright\arraybackslash}p{(\columnwidth - 4\tabcolsep) * \real{0.3276}}
  >{\raggedright\arraybackslash}p{(\columnwidth - 4\tabcolsep) * \real{0.1897}}
  >{\raggedright\arraybackslash}p{(\columnwidth - 4\tabcolsep) * \real{0.4828}}@{}}
\toprule\noalign{}
\begin{minipage}[b]{\linewidth}\raggedright
Variable
\end{minipage} & \begin{minipage}[b]{\linewidth}\raggedright
Type
\end{minipage} & \begin{minipage}[b]{\linewidth}\raggedright
Description approximative
\end{minipage} \\
\midrule\noalign{}
\endhead
\bottomrule\noalign{}
\endlastfoot
\texttt{Rk} & int64 & Rang ou ID du joueur \\
\texttt{Player} & object & Nom du joueur \\
\texttt{Nation} & object & Nationalité \\
\texttt{Pos} & object & Poste \\
\texttt{Squad} & object & Club \\
\texttt{Comp} & object & Compétition principale \\
\texttt{Age} & object & Âge \\
\texttt{Born} & object & Année de naissance \\
\ldots{} & \ldots{} & \ldots et plus de 130 autres statistiques de jeu :
buts, passes, tirs, dribbles, fautes, tacles, interceptions, passes
progressives, etc. \\
\texttt{AerWon} & float64 & Duels aériens gagnés \\
\texttt{AerLost} & float64 & Duels aériens perdus \\
\texttt{AerWon\%} & float64 & Pourcentage de duels aériens gagnés \\
\end{longtable}

\begin{center}\rule{0.5\linewidth}{0.5pt}\end{center}

\paragraph{\texorpdfstring{- 🏟️ Team Stats 2021-2022
(\texttt{2021-2022\ Football\ Team\ Stats.csv})}{- 🏟️ Team Stats 2021-2022 (2021-2022 Football Team Stats.csv)}}\label{team-stats-2021-2022-2021-2022-football-team-stats.csv}

20 variables -- chaque ligne correspond à une équipe.

\begin{longtable}[]{@{}
  >{\raggedright\arraybackslash}p{(\columnwidth - 4\tabcolsep) * \real{0.4286}}
  >{\raggedright\arraybackslash}p{(\columnwidth - 4\tabcolsep) * \real{0.2619}}
  >{\raggedright\arraybackslash}p{(\columnwidth - 4\tabcolsep) * \real{0.3095}}@{}}
\toprule\noalign{}
\begin{minipage}[b]{\linewidth}\raggedright
Variable
\end{minipage} & \begin{minipage}[b]{\linewidth}\raggedright
Type
\end{minipage} & \begin{minipage}[b]{\linewidth}\raggedright
Description
\end{minipage} \\
\midrule\noalign{}
\endhead
\bottomrule\noalign{}
\endlastfoot
\texttt{Rk} & int64 & Rang \\
\texttt{Squad} & object & Nom du club \\
\texttt{Country} & object & Pays \\
\texttt{LgRk} & int64 & Classement dans la ligue \\
\texttt{MP}, \texttt{W}, \texttt{D}, \texttt{L} & int64 & Matchs joués,
Victoires, Nuls, Défaites \\
\texttt{GF}, \texttt{GA}, \texttt{GD} & int64 & Buts pour, contre,
différence \\
\texttt{Pts}, \texttt{Pts/G} & int64 / float64 & Points et moyenne par
match \\
\texttt{xG}, \texttt{xGA}, \texttt{xGD}, \texttt{xGD/90} & float64 &
Données d'expected goals \\
\texttt{Attendance} & int64 & Affluence moyenne \\
\texttt{Top\ Team\ Scorer} & object & Meilleur buteur \\
\texttt{Goalkeeper} & object & Gardien principal \\
\end{longtable}

\begin{center}\rule{0.5\linewidth}{0.5pt}\end{center}

\paragraph{\texorpdfstring{- 🏟️ Team Stats 2022-2023
(\texttt{2022-2023\ Football\ Team\ Stats.csv})}{- 🏟️ Team Stats 2022-2023 (2022-2023 Football Team Stats.csv)}}\label{team-stats-2022-2023-2022-2023-football-team-stats.csv}

Même structure et signification que pour 2021-2022, mais avec la saison
suivante.

\begin{center}\rule{0.5\linewidth}{0.5pt}\end{center}

\paragraph{\texorpdfstring{- 🔄 Transfers Été 2022
(\texttt{2022\_2023\_football\_summer\_transfers.csv})}{- 🔄 Transfers Été 2022 (2022\_2023\_football\_summer\_transfers.csv)}}\label{transfers-uxe9tuxe9-2022-2022_2023_football_summer_transfers.csv}

11 variables -- chaque ligne correspond à un transfert.

\begin{longtable}[]{@{}
  >{\raggedright\arraybackslash}p{(\columnwidth - 4\tabcolsep) * \real{0.4211}}
  >{\raggedright\arraybackslash}p{(\columnwidth - 4\tabcolsep) * \real{0.2368}}
  >{\raggedright\arraybackslash}p{(\columnwidth - 4\tabcolsep) * \real{0.3421}}@{}}
\toprule\noalign{}
\begin{minipage}[b]{\linewidth}\raggedright
Variable
\end{minipage} & \begin{minipage}[b]{\linewidth}\raggedright
Type
\end{minipage} & \begin{minipage}[b]{\linewidth}\raggedright
Description
\end{minipage} \\
\midrule\noalign{}
\endhead
\bottomrule\noalign{}
\endlastfoot
\texttt{name} & object & Nom du joueur transféré \\
\texttt{position} & object & Poste \\
\texttt{age} & object & Âge \\
\texttt{market\_value} & object & Valeur estimée \\
\texttt{country\_from} & object & Pays de départ \\
\texttt{league\_from} & object & Ligue de départ \\
\texttt{club\_from} & object & Club de départ \\
\texttt{country\_to} & object & Pays d'arrivée \\
\texttt{league\_to} & object & Ligue d'arrivée \\
\texttt{club\_to} & object & Club d'arrivée \\
\texttt{fee} & object & Montant du transfert (peut contenir ``Free'',
``Loan'', etc.) \\
\end{longtable}

\begin{center}\rule{0.5\linewidth}{0.5pt}\end{center}

\subsection{\texorpdfstring{\textbf{Analyse et réponses aux
questions}}{Analyse et réponses aux questions}}\label{analyse-et-ruxe9ponses-aux-questions}

\begin{Shaded}
\begin{Highlighting}[]
\CommentTok{\# Chargement des packages nécessaires}

\FunctionTok{library}\NormalTok{(tidyverse)}
\FunctionTok{library}\NormalTok{(ggplot2)}
\end{Highlighting}
\end{Shaded}

\begin{Shaded}
\begin{Highlighting}[]
\CommentTok{\# Importation des données}

\NormalTok{player\_stats\_2021\_2022 }\OtherTok{\textless{}{-}} \FunctionTok{read.csv}\NormalTok{(}\StringTok{"data/2021{-}2022 Football Player Stats.csv"}\NormalTok{, }\AttributeTok{sep =} \StringTok{";"}\NormalTok{, }\AttributeTok{fileEncoding =} \StringTok{"ISO{-}8859{-}1"}\NormalTok{)}
\NormalTok{team\_stats\_2021\_2022 }\OtherTok{\textless{}{-}} \FunctionTok{read.csv}\NormalTok{(}\StringTok{"data/2021{-}2022 Football Team Stats.csv"}\NormalTok{, }\AttributeTok{sep =} \StringTok{";"}\NormalTok{, }\AttributeTok{fileEncoding =} \StringTok{"ISO{-}8859{-}1"}\NormalTok{)}
\NormalTok{team\_stats\_2022\_2023 }\OtherTok{\textless{}{-}} \FunctionTok{read.csv}\NormalTok{(}\StringTok{"data/2022{-}2023 Football Team Stats.csv"}\NormalTok{, }\AttributeTok{sep =} \StringTok{";"}\NormalTok{, }\AttributeTok{fileEncoding =} \StringTok{"ISO{-}8859{-}1"}\NormalTok{)}
\NormalTok{transfers\_2022 }\OtherTok{\textless{}{-}} \FunctionTok{read.csv}\NormalTok{(}\StringTok{"data/2022\_2023\_football\_summer\_transfers.csv"}\NormalTok{, }\AttributeTok{sep =} \StringTok{","}\NormalTok{, }\AttributeTok{fileEncoding =} \StringTok{"ISO{-}8859{-}1"}\NormalTok{)}
\end{Highlighting}
\end{Shaded}

\begin{center}\rule{0.5\linewidth}{0.5pt}\end{center}

\subsubsection{\texorpdfstring{\textbf{Question 1:} Quels clubs ont le
plus recruté par poste (top 20)
?}{Question 1: Quels clubs ont le plus recruté par poste (top 20) ?}}\label{question-1-quels-clubs-ont-le-plus-recrutuxe9-par-poste-top-20}

\paragraph{🎯 Objectif:}\label{objectif}

Identifier les \textbf{20 clubs les plus actifs} lors du mercato d'été
2022 en termes de nombre de recrues, puis visualiser quels postes ont
été ciblés en priorité par ces clubs. Cela permet de mieux comprendre
les \textbf{stratégies de renforcement} des effectifs selon les lignes
de jeu (défense, milieu, attaque).

\begin{Shaded}
\begin{Highlighting}[]
\CommentTok{\# Code}

\CommentTok{\# Nettoyage de base}
\NormalTok{transfers\_clean }\OtherTok{\textless{}{-}}\NormalTok{ transfers\_2022 }\SpecialCharTok{\%\textgreater{}\%}
  \FunctionTok{filter}\NormalTok{(}\SpecialCharTok{!}\FunctionTok{is.na}\NormalTok{(club\_to), }\SpecialCharTok{!}\FunctionTok{is.na}\NormalTok{(position), position }\SpecialCharTok{!=} \StringTok{""}\NormalTok{)}

\CommentTok{\# Regrouper et compter}
\NormalTok{recrutements\_par\_poste }\OtherTok{\textless{}{-}}\NormalTok{ transfers\_clean }\SpecialCharTok{\%\textgreater{}\%}
  \FunctionTok{group\_by}\NormalTok{(club\_to, position) }\SpecialCharTok{\%\textgreater{}\%}
  \FunctionTok{summarise}\NormalTok{(}\AttributeTok{nb\_recrues =} \FunctionTok{n}\NormalTok{(), }\AttributeTok{.groups =} \StringTok{"drop"}\NormalTok{)}

\CommentTok{\# Garder uniquement les 20 clubs ayant recruté le plus globalement}
\NormalTok{top\_clubs }\OtherTok{\textless{}{-}}\NormalTok{ recrutements\_par\_poste }\SpecialCharTok{\%\textgreater{}\%}
  \FunctionTok{group\_by}\NormalTok{(club\_to) }\SpecialCharTok{\%\textgreater{}\%}
  \FunctionTok{summarise}\NormalTok{(}\AttributeTok{total\_recrues =} \FunctionTok{sum}\NormalTok{(nb\_recrues)) }\SpecialCharTok{\%\textgreater{}\%}
  \FunctionTok{top\_n}\NormalTok{(}\DecValTok{20}\NormalTok{, total\_recrues) }\SpecialCharTok{\%\textgreater{}\%}
  \FunctionTok{pull}\NormalTok{(club\_to)}

\CommentTok{\# Filtrer les données}
\NormalTok{recrutements\_top }\OtherTok{\textless{}{-}}\NormalTok{ recrutements\_par\_poste }\SpecialCharTok{\%\textgreater{}\%}
  \FunctionTok{filter}\NormalTok{(club\_to }\SpecialCharTok{\%in\%}\NormalTok{ top\_clubs)}

\CommentTok{\# Visualisation }
\FunctionTok{ggplot}\NormalTok{(recrutements\_top, }\FunctionTok{aes}\NormalTok{(}\AttributeTok{x =}\NormalTok{ position, }\AttributeTok{y =} \FunctionTok{fct\_reorder}\NormalTok{(club\_to, nb\_recrues), }\AttributeTok{fill =}\NormalTok{ nb\_recrues)) }\SpecialCharTok{+}
  \FunctionTok{geom\_tile}\NormalTok{(}\AttributeTok{color =} \StringTok{"white"}\NormalTok{) }\SpecialCharTok{+}
  \FunctionTok{scale\_fill\_gradient}\NormalTok{(}\AttributeTok{low =} \StringTok{"lightblue"}\NormalTok{, }\AttributeTok{high =} \StringTok{"darkblue"}\NormalTok{) }\SpecialCharTok{+}
  \FunctionTok{labs}\NormalTok{(}
    \AttributeTok{title =} \StringTok{"Top 20 clubs – nombre de recrues par poste (été 2022)"}\NormalTok{,}
    \AttributeTok{x =} \StringTok{"Poste"}\NormalTok{,}
    \AttributeTok{y =} \StringTok{"Club"}\NormalTok{,}
    \AttributeTok{fill =} \StringTok{"Nombre de recrues"}
\NormalTok{  ) }\SpecialCharTok{+}
  \FunctionTok{theme\_minimal}\NormalTok{(}\AttributeTok{base\_size =} \DecValTok{12}\NormalTok{) }\SpecialCharTok{+}
  \FunctionTok{theme}\NormalTok{(}\AttributeTok{axis.text.x =} \FunctionTok{element\_text}\NormalTok{(}\AttributeTok{angle =} \DecValTok{45}\NormalTok{, }\AttributeTok{hjust =} \DecValTok{1}\NormalTok{))}
\end{Highlighting}
\end{Shaded}

\includegraphics{rapport_files/figure-latex/unnamed-chunk-1-1.pdf}

\paragraph{🧠 Interprétation du
graphique:}\label{interpruxe9tation-du-graphique}

\begin{itemize}
\item
  🔵 \textbf{US Salernitana 1919} est le club qui a le plus recruté tous
  postes confondus, avec un \textbf{focus important sur les milieux
  offensifs} et \textbf{défenseurs centraux}.
\item
  🔵 \textbf{US Lecce}, \textbf{US Cremonese} et \textbf{Udinese Calcio}
  (clubs italiens) montrent aussi une stratégie de \textbf{renforcement
  défensif}, particulièrement en \textbf{centre-back}.
\item
  🔵 \textbf{Olympique de Marseille} et \textbf{Nottingham Forest} ont
  \textbf{diversifié leurs recrutements} sur plusieurs lignes, y compris
  \textbf{les ailes} (\emph{left/right winger}).
\item
  🎯 On observe une \textbf{forte demande en milieux de terrain},
  notamment :

  \begin{itemize}
  \tightlist
  \item
    \emph{Attacking Midfield}
  \item
    \emph{Defensive Midfield}
  \item
    \emph{Centre Midfield}
  \end{itemize}
\item
  Peu de clubs ont recruté plusieurs \textbf{gardiens}, ce qui est
  logique : un club n'en fait souvent venir qu'un seul par saison.
\item
  Certains clubs comme \textbf{FC Empoli} ou \textbf{OGC Nice}
  présentent une stratégie de recrutement \textbf{équilibrée sur
  différentes lignes}, ce qui pourrait indiquer un
  \textbf{renouvellement global de l'effectif}.
\end{itemize}

\begin{center}\rule{0.5\linewidth}{0.5pt}\end{center}

\subsubsection{\texorpdfstring{\textbf{Question 2:} Quels postes sont
les plus valorisés sur le marché
?}{Question 2: Quels postes sont les plus valorisés sur le marché ?}}\label{question-2-quels-postes-sont-les-plus-valorisuxe9s-sur-le-marchuxe9}

\paragraph{🎯 Objectif:}\label{objectif-1}

Visualiser la valeur marchande des joueurs par poste pour comprendre
quels types de profils sont les plus prisés financièrement sur le
marché. Cela permet de hiérarchiser les postes selon leur importance
économique dans le football professionnel. Pour ça, on se base sur le
mercato qui a eu lieu durant l'été 2022.

\begin{Shaded}
\begin{Highlighting}[]
\CommentTok{\# Nettoyage des données}
\NormalTok{transfers\_clean }\OtherTok{\textless{}{-}}\NormalTok{ transfers\_2022 }\SpecialCharTok{\%\textgreater{}\%}
  \FunctionTok{filter}\NormalTok{(}\SpecialCharTok{!}\FunctionTok{is.na}\NormalTok{(position), }\SpecialCharTok{!}\FunctionTok{is.na}\NormalTok{(market\_value)) }\SpecialCharTok{\%\textgreater{}\%}
  \FunctionTok{distinct}\NormalTok{(name, position, age, }\AttributeTok{.keep\_all =} \ConstantTok{TRUE}\NormalTok{) }\SpecialCharTok{\%\textgreater{}\%}
  \FunctionTok{mutate}\NormalTok{(}
    \AttributeTok{market\_value\_num =} \FunctionTok{as.numeric}\NormalTok{(market\_value)  }\CommentTok{\# Convertir les valeurs en numérique}
\NormalTok{  ) }\SpecialCharTok{\%\textgreater{}\%}
  \FunctionTok{filter}\NormalTok{(market\_value\_num }\SpecialCharTok{\textgreater{}=} \FloatTok{1.5}\NormalTok{)  }\CommentTok{\# Filtrer les valeurs négatives/nulles ou pas réalistes genre en dessous de 1.5 millions}
\end{Highlighting}
\end{Shaded}

\begin{verbatim}
## Warning: There was 1 warning in `mutate()`.
## i In argument: `market_value_num = as.numeric(market_value)`.
## Caused by warning:
## ! NAs introduits lors de la conversion automatique
\end{verbatim}

\begin{Shaded}
\begin{Highlighting}[]
\CommentTok{\# On crée une colonne pour regrouper certains postes qui sont très proches}
\NormalTok{transfers\_clean }\OtherTok{\textless{}{-}}\NormalTok{ transfers\_clean }\SpecialCharTok{\%\textgreater{}\%}
  \FunctionTok{mutate}\NormalTok{(}\AttributeTok{position\_fr =}\NormalTok{ position) }\SpecialCharTok{\%\textgreater{}\%}
  \FunctionTok{mutate}\NormalTok{(}\AttributeTok{position\_fr =} \FunctionTok{recode}\NormalTok{(position\_fr,}
                              \StringTok{"Second Striker"} \OtherTok{=} \StringTok{"Attaquant"}\NormalTok{,}
                              \StringTok{"Centre{-}Forward"} \OtherTok{=} \StringTok{"Attaquant"}\NormalTok{,}
                              \StringTok{"Forward Attacker"} \OtherTok{=} \StringTok{"Attaquant"}\NormalTok{,}
                              \StringTok{"attack"} \OtherTok{=} \StringTok{"Attaquant"}\NormalTok{,}
                              \StringTok{"Right Midfield"} \OtherTok{=} \StringTok{"Ailier droit"}\NormalTok{,}
                              \StringTok{"Left Midfield"} \OtherTok{=} \StringTok{"Ailier gauche"}\NormalTok{,}
                              \StringTok{"Attacking Midfield"} \OtherTok{=} \StringTok{"Milieu offensif"}\NormalTok{,}
                              \StringTok{"Central Midfield"} \OtherTok{=} \StringTok{"Milieu central"}\NormalTok{,}
                              \StringTok{"Defensive Midfield"} \OtherTok{=} \StringTok{"Milieu central"}\NormalTok{, }
                              \StringTok{"Centre{-}Back"} \OtherTok{=} \StringTok{"Défenseur central"}\NormalTok{,}
                              \StringTok{"defence"} \OtherTok{=} \StringTok{"Défenseur central"}\NormalTok{,}
                              \StringTok{"Right Winger"} \OtherTok{=} \StringTok{"Ailier droit"}\NormalTok{,}
                              \StringTok{"Left Winger"} \OtherTok{=} \StringTok{"Ailier gauche"}\NormalTok{, }
                              \StringTok{"Left{-}Back"} \OtherTok{=} \StringTok{"Défenseur gauche"}\NormalTok{,}
                              \StringTok{"Right{-}Back"} \OtherTok{=} \StringTok{"Défenseur droit"}\NormalTok{,}
                              \StringTok{"Goalkeeper"} \OtherTok{=} \StringTok{"Gardien de but"}\NormalTok{,}
                              \AttributeTok{.default =}\NormalTok{ position\_fr))  }\CommentTok{\# Par défaut, on garde la position d\textquotesingle{}origine}

\CommentTok{\# Calcul de la moyenne et de l\textquotesingle{}écart{-}type des valeurs marchandes par poste}
\NormalTok{transfers\_clean\_stats }\OtherTok{\textless{}{-}}\NormalTok{ transfers\_clean }\SpecialCharTok{\%\textgreater{}\%}
  \FunctionTok{group\_by}\NormalTok{(position\_fr) }\SpecialCharTok{\%\textgreater{}\%}
  \FunctionTok{filter}\NormalTok{(}\FunctionTok{n}\NormalTok{() }\SpecialCharTok{\textgreater{}=} \DecValTok{5}\NormalTok{) }\SpecialCharTok{\%\textgreater{}\%}  
  \FunctionTok{summarise}\NormalTok{(}
    \AttributeTok{mean\_value =} \FunctionTok{mean}\NormalTok{(market\_value\_num, }\AttributeTok{na.rm =} \ConstantTok{TRUE}\NormalTok{),}
    \AttributeTok{sd\_value =} \FunctionTok{sd}\NormalTok{(market\_value\_num, }\AttributeTok{na.rm =} \ConstantTok{TRUE}\NormalTok{)}
\NormalTok{  )}

\CommentTok{\# Visualisation avec un barplot et des barres d\textquotesingle{}erreur pour voir les postes qui ont une grosse différence de prix}
\FunctionTok{ggplot}\NormalTok{(transfers\_clean\_stats, }\FunctionTok{aes}\NormalTok{(}\AttributeTok{x =} \FunctionTok{reorder}\NormalTok{(position\_fr, mean\_value), }\AttributeTok{y =}\NormalTok{ mean\_value)) }\SpecialCharTok{+}
  \FunctionTok{geom\_bar}\NormalTok{(}\AttributeTok{stat =} \StringTok{"identity"}\NormalTok{, }\AttributeTok{fill =} \StringTok{"\#4CAF50"}\NormalTok{) }\SpecialCharTok{+}  
  \FunctionTok{geom\_errorbar}\NormalTok{(}\FunctionTok{aes}\NormalTok{(}\AttributeTok{ymin =} \FunctionTok{pmax}\NormalTok{(mean\_value }\SpecialCharTok{{-}}\NormalTok{ sd\_value, }\DecValTok{0}\NormalTok{), }\AttributeTok{ymax =}\NormalTok{ mean\_value }\SpecialCharTok{+}\NormalTok{ sd\_value), }\AttributeTok{width =} \FloatTok{0.2}\NormalTok{) }\SpecialCharTok{+}
  \FunctionTok{labs}\NormalTok{(}
    \AttributeTok{title =} \StringTok{"Valeur marchande moyenne par poste (mercato été 2022)"}\NormalTok{,}
    \AttributeTok{x =} \StringTok{"Poste"}\NormalTok{,}
    \AttributeTok{y =} \StringTok{"Valeur marchande moyenne (Millions d\textquotesingle{}€)"}
\NormalTok{  ) }\SpecialCharTok{+}
  \FunctionTok{theme\_minimal}\NormalTok{(}\AttributeTok{base\_size =} \DecValTok{12}\NormalTok{) }\SpecialCharTok{+}
  \FunctionTok{coord\_flip}\NormalTok{()}
\end{Highlighting}
\end{Shaded}

\includegraphics{rapport_files/figure-latex/unnamed-chunk-2-1.pdf}

\begin{Shaded}
\begin{Highlighting}[]
\NormalTok{transfers\_clean }\SpecialCharTok{\%\textgreater{}\%}
  \FunctionTok{count}\NormalTok{(position\_fr, }\AttributeTok{sort =} \ConstantTok{TRUE}\NormalTok{)}
\end{Highlighting}
\end{Shaded}

\begin{verbatim}
##         position_fr   n
## 1    Milieu central 271
## 2         Attaquant 245
## 3 Défenseur central 203
## 4     Ailier gauche 121
## 5      Ailier droit  96
## 6   Milieu offensif  95
## 7  Défenseur gauche  82
## 8   Défenseur droit  76
## 9    Gardien de but  63
\end{verbatim}

\begin{Shaded}
\begin{Highlighting}[]
\FunctionTok{print}\NormalTok{(transfers\_clean }\SpecialCharTok{\%\textgreater{}\%} \FunctionTok{count}\NormalTok{(position\_fr, }\AttributeTok{sort =} \ConstantTok{TRUE}\NormalTok{))}
\end{Highlighting}
\end{Shaded}

\begin{verbatim}
##         position_fr   n
## 1    Milieu central 271
## 2         Attaquant 245
## 3 Défenseur central 203
## 4     Ailier gauche 121
## 5      Ailier droit  96
## 6   Milieu offensif  95
## 7  Défenseur gauche  82
## 8   Défenseur droit  76
## 9    Gardien de but  63
\end{verbatim}

\paragraph{🧠 Interprétation du
graphique:}\label{interpruxe9tation-du-graphique-1}

Attaquant et Buteur : Les attaquants sont les joueurs les plus valorisés
sur le marché, avec une valeur moyenne qui dépasse largement celle des
autres postes. Cela reflète l'importance des joueurs offensifs dans le
football moderne, où les buts et les performances offensives sont
souvent décisives mais surtout les attaquants sont souvent les joueurs
qui attirent le plus l'attention des fans et des médias, ce qui en fait
des atouts marketing précieux pour les clubs et qui explique leur valeur
bien plus élevé

Défenseur central : Les défenseurs centraux occupent la deuxième place
en termes de valeur marchande. Leur rôle crucial dans la stabilité
défensive et leur capacité à organiser la défense expliquent cette
valorisation élevée. Un bon défenseur central peut être la clé de voûte
d'une équipe solide et équilibrée.

Ailier droit et Ailier gauche : Les ailiers, qu'ils soient droits ou
gauches, présentent une valeur marchande significative sur le marché des
transferts avec respectivement la 3ème et 5ème place. Toutefois, on
observe souvent une légère survalorisation des ailiers droits. Cette
différence peut s'expliquer par la rareté relative des ailiers droits
car cela implique souvent d'être gaucher, pour être capables de repiquer
dans l'axe et de frapper avec leur pied fort. Ce profil est très
recherché dans le football moderne d'où une valorisation plus élevée.

Milieu central et offensif : Les milieux centraux, souvent considérés
comme les ``poumons'' de l'équipe, sont également très valorisés. Leur
polyvalence et leur capacité à influencer le jeu dans les deux sens en
font des acteurs clés sur le marché. Paradoxalement, les milieux
offensifs semblent être moins bien valorisés. C'est un poste qui, dans
le football moderne, a perdu de sa splendeur par rapport à avant et
c'est caractérisé par une baisse de la valeur marchande à ce poste.

Défenseur gauche et Défenseur droit : Les défenseurs latéraux, bien que
moins valorisés que les défenseurs centraux, restent importants.Ils sont
très recherchés dans le football moderne car ils doivent être
polyvalents, capables de défendre et d'attaquer efficacement.On ne
retrouve pas la rareté des latéraux droit par rapport au gauche comme on
peut avoir avec les ailiers et on remarque qu'à l'inverse les latéraux
gauche on tendance à avoir une plus grosse valeur marchande.

Gardien de but : Les gardiens de but sont nettement moins valorisés que
les autres postes. Cela s'explique en partie par la spécificité de ce
rôle, qui diffère fortement des autres positions sur le terrain. Par
extension, c'est aussi le poste le moins ``vendeur'' du football, ce qui
se reflète dans leur valeur marchande.

\begin{center}\rule{0.5\linewidth}{0.5pt}\end{center}

\subsubsection{\texorpdfstring{\textbf{Question 3:} Corrélation entre xG
et points
?}{Question 3: Corrélation entre xG et points ?}}\label{question-3-corruxe9lation-entre-xg-et-points}

\paragraph{🎯 Objectif:}\label{objectif-2}

Analyser si le nombre d'expected goals (xG) réalisés par une équipe est
corrélé avec son total de points en championnat. Cela permet de vérifier
si produire beaucoup d'occasions (même sans forcément marquer) est un
bon indicateur de performance globale. On utilise les données de la
saison 2022-2023.

\begin{Shaded}
\begin{Highlighting}[]
\CommentTok{\# On utilise les données de la saison 2022{-}2023}
\NormalTok{team\_stats\_22\_23 }\OtherTok{\textless{}{-}}\NormalTok{ team\_stats\_2022\_2023 }\SpecialCharTok{\%\textgreater{}\%}
  \FunctionTok{select}\NormalTok{(Squad, xG, Pts)}

\CommentTok{\# Nettoyage basique : on supprime les NA au cas où}
\NormalTok{team\_stats\_22\_23 }\OtherTok{\textless{}{-}}\NormalTok{ team\_stats\_22\_23 }\SpecialCharTok{\%\textgreater{}\%}
  \FunctionTok{filter}\NormalTok{(}\SpecialCharTok{!}\FunctionTok{is.na}\NormalTok{(xG), }\SpecialCharTok{!}\FunctionTok{is.na}\NormalTok{(Pts))}

\CommentTok{\# Visualisation scatter plot}
\FunctionTok{ggplot}\NormalTok{(team\_stats\_22\_23, }\FunctionTok{aes}\NormalTok{(}\AttributeTok{x =}\NormalTok{ xG, }\AttributeTok{y =}\NormalTok{ Pts)) }\SpecialCharTok{+}
  \FunctionTok{geom\_point}\NormalTok{(}\AttributeTok{color =} \StringTok{"\#1f77b4"}\NormalTok{, }\AttributeTok{size =} \DecValTok{3}\NormalTok{) }\SpecialCharTok{+}
  \FunctionTok{geom\_smooth}\NormalTok{(}\AttributeTok{method =} \StringTok{"lm"}\NormalTok{, }\AttributeTok{se =} \ConstantTok{FALSE}\NormalTok{, }\AttributeTok{color =} \StringTok{"red"}\NormalTok{, }\AttributeTok{linetype =} \StringTok{"dashed"}\NormalTok{) }\SpecialCharTok{+}
  \FunctionTok{labs}\NormalTok{(}
    \AttributeTok{title =} \StringTok{"Corrélation entre Expected Goals (xG) et Points en championnat (2022{-}23)"}\NormalTok{,}
    \AttributeTok{x =} \StringTok{"Expected Goals (xG)"}\NormalTok{,}
    \AttributeTok{y =} \StringTok{"Points"}\NormalTok{,}
    \AttributeTok{caption =} \StringTok{"Source : 2022{-}2023 Football Team Stats.csv"}
\NormalTok{  ) }\SpecialCharTok{+}
  \FunctionTok{theme\_minimal}\NormalTok{(}\AttributeTok{base\_size =} \DecValTok{12}\NormalTok{)}
\end{Highlighting}
\end{Shaded}

\begin{verbatim}
## `geom_smooth()` using formula = 'y ~ x'
\end{verbatim}

\includegraphics{rapport_files/figure-latex/unnamed-chunk-3-1.pdf}

\begin{Shaded}
\begin{Highlighting}[]
\CommentTok{\# Calcul du coefficient de corrélation de Pearson}
\NormalTok{cor\_xg\_pts }\OtherTok{\textless{}{-}} \FunctionTok{cor}\NormalTok{(team\_stats\_22\_23}\SpecialCharTok{$}\NormalTok{xG, team\_stats\_22\_23}\SpecialCharTok{$}\NormalTok{Pts, }\AttributeTok{use =} \StringTok{"complete.obs"}\NormalTok{)}
\FunctionTok{paste0}\NormalTok{(}\StringTok{"Coefficient de corrélation de Pearson : "}\NormalTok{, }\FunctionTok{round}\NormalTok{(cor\_xg\_pts, }\DecValTok{3}\NormalTok{))}
\end{Highlighting}
\end{Shaded}

\begin{verbatim}
## [1] "Coefficient de corrélation de Pearson : 0.815"
\end{verbatim}

\paragraph{🧠 Interprétation du
graphique:}\label{interpruxe9tation-du-graphique-2}

Le graphique montre qu'il existe une forte corrélation positive entre
les expected goals (xG) et les points obtenus en championnat lors de la
saison 2022-2023. Le coefficient de corrélation de Pearson est de 0,815,
ce qui indique qu'en général, plus une équipe génère d'occasions de but
de qualité (mesurées par les xG), plus elle obtient de points au
classement. Cela confirme que produire beaucoup d'occasions est un
élément important pour réussir sur toute une saison.

Cependant, la relation n'est pas parfaite. On observe que certains
points sont assez éloignés de la tendance générale. Cela peut
s'expliquer par plusieurs facteurs, comme l'efficacité offensive,
c'est-à-dire la capacité à transformer les occasions en buts, ou encore
la solidité défensive. Une équipe peut créer beaucoup d'occasions mais
manquer de réalisme ou encaisser trop de buts, ce qui limite son total
de points.

En résumé, les expected goals sont un bon indicateur des performances
d'une équipe, mais ils ne suffisent pas à eux seuls pour expliquer
complètement les résultats. D'autres aspects du jeu, comme la finition
ou la défense, restent essentiels.

\begin{center}\rule{0.5\linewidth}{0.5pt}\end{center}

\subsubsection{\texorpdfstring{\textbf{Question 4:} Quel est l'impact
des transferts sur la performance des clubs entre 2021-2022 et 2022-2023
?}{Question 4: Quel est l'impact des transferts sur la performance des clubs entre 2021-2022 et 2022-2023 ?}}\label{question-4-quel-est-limpact-des-transferts-sur-la-performance-des-clubs-entre-2021-2022-et-2022-2023}

\paragraph{🎯 Objectif:}\label{objectif-3}

Analyser l'évolution de la performance des clubs entre deux saisons
(2021-2022 ➔ 2022-2023) en fonction de leur nombre de recrues durant
l'été 2022.

L'objectif est de voir s'il existe une corrélation entre l'activité sur
le marché des transferts (quantité de recrutements) et l'évolution du
classement d'une saison sur l'autre.

\begin{Shaded}
\begin{Highlighting}[]
\CommentTok{\# Regrouper les clubs qui ont recruté et compter}
\NormalTok{nb\_recrues\_club }\OtherTok{\textless{}{-}}\NormalTok{ transfers\_clean }\SpecialCharTok{\%\textgreater{}\%}
  \FunctionTok{group\_by}\NormalTok{(club\_to) }\SpecialCharTok{\%\textgreater{}\%}
  \FunctionTok{summarise}\NormalTok{(}\AttributeTok{nb\_recrues =} \FunctionTok{n}\NormalTok{(), }\AttributeTok{.groups =} \StringTok{"drop"}\NormalTok{)}

\CommentTok{\# Préparer les classements par saison}
\NormalTok{classements\_2021\_2022 }\OtherTok{\textless{}{-}}\NormalTok{ team\_stats\_2021\_2022 }\SpecialCharTok{\%\textgreater{}\%}
  \FunctionTok{select}\NormalTok{(Squad, LgRk) }\SpecialCharTok{\%\textgreater{}\%}
  \FunctionTok{rename}\NormalTok{(}\AttributeTok{club =}\NormalTok{ Squad, }\AttributeTok{classement\_2022 =}\NormalTok{ LgRk)}

\NormalTok{classements\_2022\_2023 }\OtherTok{\textless{}{-}}\NormalTok{ team\_stats\_2022\_2023 }\SpecialCharTok{\%\textgreater{}\%}
  \FunctionTok{select}\NormalTok{(Squad, LgRk) }\SpecialCharTok{\%\textgreater{}\%}
  \FunctionTok{rename}\NormalTok{(}\AttributeTok{club =}\NormalTok{ Squad, }\AttributeTok{classement\_2023 =}\NormalTok{ LgRk)}

\CommentTok{\# Fusionner tout ensemble}
\NormalTok{evolution\_clubs }\OtherTok{\textless{}{-}}\NormalTok{ classements\_2021\_2022 }\SpecialCharTok{\%\textgreater{}\%}
  \FunctionTok{inner\_join}\NormalTok{(classements\_2022\_2023, }\AttributeTok{by =} \StringTok{"club"}\NormalTok{) }\SpecialCharTok{\%\textgreater{}\%}
  \FunctionTok{left\_join}\NormalTok{(nb\_recrues\_club, }\AttributeTok{by =} \FunctionTok{c}\NormalTok{(}\StringTok{"club"} \OtherTok{=} \StringTok{"club\_to"}\NormalTok{)) }\SpecialCharTok{\%\textgreater{}\%}
  \FunctionTok{mutate}\NormalTok{(}
    \AttributeTok{nb\_recrues =} \FunctionTok{replace\_na}\NormalTok{(nb\_recrues, }\DecValTok{0}\NormalTok{),  }\CommentTok{\# certains clubs peuvent avoir 0 recrutement}
    \AttributeTok{evolution\_classement =}\NormalTok{ classement\_2022 }\SpecialCharTok{{-}}\NormalTok{ classement\_2023}
\NormalTok{  )}

\CommentTok{\# Pour la visualisation, créer des classes selon nombre de recrues}
\NormalTok{evolution\_clubs }\OtherTok{\textless{}{-}}\NormalTok{ evolution\_clubs }\SpecialCharTok{\%\textgreater{}\%}
  \FunctionTok{mutate}\NormalTok{(}
    \AttributeTok{categorie\_recrutement =} \FunctionTok{case\_when}\NormalTok{(}
\NormalTok{      nb\_recrues }\SpecialCharTok{==} \DecValTok{0} \SpecialCharTok{\textasciitilde{}} \StringTok{"0 recrue"}\NormalTok{,}
\NormalTok{      nb\_recrues }\SpecialCharTok{\textless{}=} \DecValTok{3} \SpecialCharTok{\textasciitilde{}} \StringTok{"1{-}3 recrues"}\NormalTok{,}
\NormalTok{      nb\_recrues }\SpecialCharTok{\textless{}=} \DecValTok{6} \SpecialCharTok{\textasciitilde{}} \StringTok{"4{-}6 recrues"}\NormalTok{,}
      \ConstantTok{TRUE} \SpecialCharTok{\textasciitilde{}} \StringTok{"7+ recrues"}
\NormalTok{    )}
\NormalTok{  )}

\CommentTok{\# Visualisation : boxplot pour voir la distribution de l\textquotesingle{}évolution du classement par catégorie de recrutement}
\FunctionTok{ggplot}\NormalTok{(evolution\_clubs, }\FunctionTok{aes}\NormalTok{(}\AttributeTok{x =}\NormalTok{ categorie\_recrutement, }\AttributeTok{y =}\NormalTok{ evolution\_classement, }\AttributeTok{fill =}\NormalTok{ categorie\_recrutement)) }\SpecialCharTok{+}
  \FunctionTok{geom\_boxplot}\NormalTok{(}\AttributeTok{alpha =} \FloatTok{0.7}\NormalTok{) }\SpecialCharTok{+}
  \FunctionTok{scale\_fill\_manual}\NormalTok{(}\AttributeTok{values =} \FunctionTok{c}\NormalTok{(}\StringTok{"lightblue"}\NormalTok{, }\StringTok{"skyblue"}\NormalTok{, }\StringTok{"blue"}\NormalTok{, }\StringTok{"darkblue"}\NormalTok{)) }\SpecialCharTok{+}
  \FunctionTok{geom\_hline}\NormalTok{(}\AttributeTok{yintercept =} \DecValTok{0}\NormalTok{, }\AttributeTok{linetype =} \StringTok{"dashed"}\NormalTok{, }\AttributeTok{color =} \StringTok{"red"}\NormalTok{) }\SpecialCharTok{+}
  \FunctionTok{labs}\NormalTok{(}
    \AttributeTok{title =} \StringTok{"Impact du recrutement sur l\textquotesingle{}évolution du classement (2021{-}2023)"}\NormalTok{,}
    \AttributeTok{subtitle =} \StringTok{"Distribution de l\textquotesingle{}évolution de position en fonction du nombre de recrues estivales"}\NormalTok{,}
    \AttributeTok{x =} \StringTok{"Nombre de recrues (catégories)"}\NormalTok{,}
    \AttributeTok{y =} \StringTok{"Evolution du classement (positif = amélioration)"}\NormalTok{,}
    \AttributeTok{fill =} \StringTok{"Recrutements"}
\NormalTok{  ) }\SpecialCharTok{+}
  \FunctionTok{theme\_minimal}\NormalTok{(}\AttributeTok{base\_size =} \DecValTok{12}\NormalTok{)}
\end{Highlighting}
\end{Shaded}

\includegraphics{rapport_files/figure-latex/unnamed-chunk-4-1.pdf}

\paragraph{🧠 Interprétation du
graphique:}\label{interpruxe9tation-du-graphique-3}

Le graphique montre la relation entre le nombre de joueurs recrutés par
un club durant l'été 2022 et l'évolution de son classement entre les
saisons 2021-2022 et 2022-2023. L'évolution du classement est calculée
en faisant le classement de la saison 2022 moins celui de la saison
2023. Un résultat positif signifie que le club s'est amélioré (il est
monté dans le classement), tandis qu'un résultat négatif indique une
baisse de performance.

Globalement dans ce graphique, on observe que :

Les clubs qui n'ont recruté aucun joueur ont eu des résultats très
variables,ils affichent une médiane proche de zéro, donc peu d'impact.

Les clubs qui ont recruté entre 1 et 3 joueurs ont plutôt vu leur
classement se dégrader (médiane en dessous de 0).

Les clubs ayant fait 4 à 6 recrues s'en sortent plutôt bien, avec une
petite amélioration en moyenne et une médiane légérement supérieure à 0.

Enfin, pour les clubs ayant recruté 7 joueurs ou plus, les résultats
sont vraiment très variés : certains ont beaucoup progressé, d'autres
ont fortement régressé. La médiane est plutôt négative, ce qui montre
que recruter en masse ne garantit pas une amélioration du classement.

En résumé, le graphique montre qu'il n'existe pas de lien clair entre le
nombre de recrues et une amélioration du classement. Recruter beaucoup
n'est pas toujours efficace, et une stratégie plus ciblée (recruter
moins mais mieux) pourrait être plus rentable pour progresser.

\end{document}
